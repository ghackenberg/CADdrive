\title{GitHub for CAD -- How could that look like?}

\author{
    \IEEEauthorblockN{Dominik Frühwirth}
    \IEEEauthorblockA{
        School of Engineering, University of \\
        Applied Sciences Upper Austria \\
        4600 Wels, Upper Austria, Austria \\
        \href{mailto:dominik.fruewirth@fh-wels.at}{dominik.fruewirth@fh-wels.at}
    }
    \and
    \IEEEauthorblockN{Georg Hackenberg}
    \IEEEauthorblockA{
        School of Engineering, University of \\
        Applied Sciences Upper Austria \\
        4600 Wels, Upper Austria, Austria \\
        \href{mailto:georg.hackenberg@fh-wels.at}{georg.hackenberg@fh-wels.at}
    }
    \and
    \IEEEauthorblockN{Christian Zehetner}
    \IEEEauthorblockA{
        School of Engineering, University of \\
        Applied Sciences Upper Austria \\
        4600 Wels, Upper Austria, Austria \\
        \href{mailto:christian.zehetner@fh-wels.at}{christian.zehetner@fh-wels.at}
    }
}

\maketitle

\begin{abstract}
    Product development is facing new challenges as a result of the demand for increasingly complex and individualized products in small batch sizes and short time to markets. 
    By providing software tools, digitalization is an enabler for agile product development.
    Stakeholders such as customers, project managers, requirement engineers and product designers can develop the product together and react quickly to each other's demands.
    Today, each stakeholder uses his/her own platform to manage artifacts like product designs, requirements or schedules.
    The array of platforms complicates the information flow and frequently yields inaccuracies and inconsistencies.
    To solve these issues, we have developed an open source platform on which all relevant results can be managed.
    Our platform provides the means for storing product design revisions as well as design tasks and project schedules through an integrated data model.
    We complement this data model with a role-based permission model restricting the changes each stakeholder can perform to the underlying data model.
    Finally, we propose an interface model through which the end users can access the design, task, and schedule data and execute the available operations.
\end{abstract}

\begin{IEEEkeywords}
    Product development, computer-aided design, requirements engineering, project management
\end{IEEEkeywords}