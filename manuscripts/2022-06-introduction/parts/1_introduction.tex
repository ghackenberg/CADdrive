\section{Introduction}
\label{sec:introduction}

\cite{Ahti2005} already noted that customers across industries demand increasingly complex and individualized products with higher quality and shorter time to markets.
At the same time, engineers need to deal with (partially) vague requirements specifications and customers expect flexibility with respect to change requests -- all due to uncertainties that cannot be resolved at the beginning of product development.
\cite{6226784} show that systematic requirements engineering with early stakeholder involvement and dedicated tool support represents half of the success of future product development.
To meet these challenges, amongst others \cite{ozkan2019agile} propose agile methods which increase the chances of delivering products with better quality in shorter time and with lower cost. 
\cite{HEIMICKE2021786} argue that the majority of agile methods originate from the domain of software development, where traditional development methods frequently led to unsuccessful projects.
In this domain, today established platforms exist -- such as GitHub\footnote{\url{https://www.github.com/}} -- which provide dedicated support for agile projects with high stakeholder involvement based on an integrated view on requirements, schedules, and deliverables.
Unfortunately, due to the nature of the deliverables (mainly source code, binaries, and documentation), these platforms cannot be applied to product development directly. 
\cite{MarionTucker} explain that during the development of physical products, typically various independent software tools are used for different activities, such as project management, requirements engineering, and product design.
\cite{houshmand2010collaborative} deduct that due to the independence of these software tools, isolated artifacts are generated, which lack a common and integrated data model.
\cite{Jorma2014} show that the lack of a common and integrated data model, in turn, hinders the information flow between stakeholders and can cause inaccuracies and inconsistencies.
These inaccuracies and inconsistencies finally lead to inefficiencies in product development, which cause unnecessary cost and project delays.
This observation leads us to the following research questions.

\subsection{Research questions}

How can we improve the information flow between customers, project managers, requirements engineers, and product designers in agile product development?
How can we better integrate the available information about requirements, schedules, and deliverables in agile product development?
How can we maybe translate the ideas from established software development platforms such as GitHub to the domain of product development?
Which elements of these platforms can we reuse and which elements do we need to adapt to support product development activities directly?
To answer these questions, we worked on a number of scientific contributions.

\subsection{Scientific contributions}

We first provide an integrated \textbf{data model} of requirements, schedules, and deliverables, which supports management of CAD model revisions, linking of requirements to parts and assemblies of CAD model revisions, asynchronous discussion about and clarification of requirements, as well as prioritization and scheduling of requirements through milestones with fixed start and end dates.
Additionally, we provide an \textbf{interface model}, which we derived from GitHub, but which required substantial changes to support the underlying data model, corresponding operations, as well as individual stakeholder perspectives properly.
In this article we describe and discuss both contributions as outlined in the following.

\subsection{Document contents}

In Section~\ref{sec:differentiation} we first discuss related work on requirements engineering and data integration in the context of agile product development.
Then, in Section~\ref{sec:requirements} we summarize the requirements we believe an integration approach to project management, requirements engineering, and product design must fulfill.
Thereafter, in Section~\ref{sec:contribution} we present the integrated data model and the Web-based interface model as explained in the previous section.
Afterwards, in Section~\ref{sec:evaluation} we evaluate our models with respect to different criteria and unveil weaknesses, which need to be addressed eventually to be widely applicable.
Finally, in Section~\ref{sec:conclusion} we summarize our learnings and provide an outlook on future work.