\subsection{Permission model}

\label{subsec:permissionModel}
The permission model offers a number of possible restrictions on the platform so that not every user has all freedoms. This is especially important when cooperating with customers. The customer should only have the possibility to evaluate existing products. For this he can see the products he is registered for and use the given product management functions depending on its member role \textit{customer}. Creating new users and products should only be possible by the managers of the respective product. They also organize the rights of each user. The permission model is divided into two levels. The first level is the global level and the second level is the product level. The global level uses a permission model and the product level a role model which is implicitly linked to the permission model. 
This allows the first level to manage products and users and the role model to distribute more specific permissions to the individual products. This system is deeply integrated in the backend.

\subsubsection*{Global level}

The \textit{User} entity has the two attributes \textit{user management permission} and \textit{product management permission}. Only a user who has the user management permission can create or edit users. On the other hand every user can edit his own personal data. With the product management permission it is possible to create new products. If a new product is created, the respective user is automatically also a product member and receives the member role \textit{manager}. The permissions of each user can be adjusted afterwards. So it is possible to give multiple users the permissions for user management and product management. For example, a user and product manager can exist for each department. 

\subsubsection*{Product level}

For permission management at product level, three member roles are provided. These roles are: \textit{manager, engineer and customer}. The manager is the one who created the product and has the permission to add more members. So he can add more managers, who in turn have all the rights over the management of the product. This is useful when the product development covers several departments. The second role is the engineer. He is involved in the product development process. He has no rights to change the product description or its members. However, he can freely create and edit versions, issues and comments. The last role is the customer who has the possibility to observe the product development process. He can create issues and comments and so participate in the product development process. In the current version of the software the rights of the customer are still very strict. The permission system is implemented in such a way that it can be changed with few adjustments. If the customer needs writing permissions, this can be easily changed.