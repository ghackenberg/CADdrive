\section{Tool evaluation}
\label{sec:evaluation}

To solve the issues defined at the beginning of the article, we have developed the software ProductBoard. The software has not yet reached its final state but already meets a wide range of requirement. The platform \textit{GitHub} plays an important role as a reference for the evaluation for our tool.
We have noticed a few points that are still insufficiently covered in the current software version or are still open questions.

\subsection*{Support for multiple file types}

The view on the right side of the user interface can theoretically be used for all kinds of data.
At this stage, the tool only supports 3D CAD files in GLB format. To be able to support other disciplines than the design of 3d CAD models, the software needs to be modified to read and display exports in different formats. The software can also be extended to include image files, video files, or simulations like FEM, CFD, MBS and DES. Videos and simulations of systems could be integrated directly into the tool and controlled there.
Due to the extension to different data formats, our software can be used for the most diverse development processes. 
2D plans can be used for example for construction drawings, electrical drawings and building floor plans to support mechanical engineering, electrical engineering or architects in various domains.
By sharing images, videos and documents, the platform can be used even more flexibly to support product development.
If different data formats are supported, it should also be possible to mark parts in these data for disscussions.
For example, it should be possible to mark image areas in 2D formats such as construction drawings, circuit diagrams, PDFs or images. 
In audio and video files, it should be possible to mark time segments.

\subsection*{Bind management tools on products or versions}

During the test phase, we asked ourselves whether it is better for mechanical designs to tie the issues and comments to the respective product or to a product version. Like the GitHub platform, our implementation binds the issues and comments to the respective product. This has the advantage that there is a better overall view of the product and the product can be managed version-independently. It allows issues to be defined before a version is created. This means that requirements that will be implemented later can already be included before the project begins. In addition, it may be that the implementation of a requirement extends over several versions. This case can be handled much easier if issues and comments are attached to the product and not to a specific product version.
Because of these advantages, we have chosen this general approach. Linking issues and versions is still possible and is more a matter of representation in the UI, than a issue regarding the data model

\subsection*{Referencing components across multiple versions}

In the comments section, different parts of the CAD model can be referenced in text and get highlighted. This part references are always bound to a version of a product. The problem is that it is possible that in the course of a discussion, parts of different versions are referenced because the product changes in the course of the project. In the communication channel, however, only one version of the product is visible and thus only the parts that have been referenced in this version can be seen. In the test phase it turned out that this can be improved. Here an overview must be built in the user interface, which can display all versions with the respective references so that the progress of the product is traceable. Another solution would be to transfer component markings to new versions semi-automatically. So that the user gets suggestions and confirms them or picks a selection himself.

\subsection*{3D view loads always the latest version}

If you switch between the different views in the user interface, for example from the issues to the milestones, the latest product version is always loaded and displayed in the 3D view on the right side. Therefore, in each view the version to be displayed must again be selected separately via the dropdown menu. This issue has a negative impact on the user experience and a solution must be found.
A solution for this issue would be to store the current version in memory or to change the structure of the UI.

\subsection*{Rigid permission model cause problems}

The primitive role system that is currently implemented at the member level is not suitable for complex organizational structures. Here, the permission model and the data model must be modified so that the permissions can be fine-tuned. We see either the approach with several fixed member roles or with single permissions like \textit{can open or close Issues, can create comments, can edit milestones} and so on, which can be assigned to the respective member. In the user interface, the respective area must then also be designed so that the rights can be distributed more individually. In our opinion, the permissions on the user level are sufficient to be able to create users and products because the fine-tuning of the permissions then takes place via the member level.

\subsection*{The UI is not optimized for mobile devices}

The platform is currently designed exclusively for large screens and not intended as a mobile application, which is a major weakness. For this purpose, the user interface must be changed to a responsive design to look good on all devices and to support the heterogeneous variety of devices across all stakeholders. If the software becomes popular in the world of product development, a change to a responsive design will be necessary. For this purpose the user interface will have to be adapted to all possible end devices.

\subsection*{Asynchronous versus synchronous collaboration}

Currently, asynchronous collaboration like on GitHub is possible using the tool. Setting up a synchronous collaboration within a company structure including the customer would bring many advantages, disadvantages and development effort with it. For synchronous collaboration, for example, care would have to be taken to ensure that user experience does not suffer if new content is constantly popping up at any point. Currently, the software is designed for asynchronous collaboration and whether a change will take place will be evaluated in the future.

\subsection*{Test cases for requirements}

The requirement management is implemented in our tool via the issues and comments and enables the requirements to be presented in text form. An extension for the requirement management would be the implementation of automated test cases. By formalizing the requirements, test cases can be generated that automatically check for the fulfillment of the issues. Thus, the issues can be automatically closed or reopened depending on the test results.

\subsection*{Further functional enhancements}

\subsubsection*{Product audit and product review}

% Das kann auch synchron oder asynchron ablaufen und mit jedem Endgerät (auch VR).
In the current version of the software is still missing a possibility to audit or revise the products. For this purpose, a section can be implemented in the tool to perform audits and reviews. 
They should be performable on any end device and virtual reality also offers a good possibility. Audits and reviews should be able to run either asynchronously or synchronously.

\subsubsection*{Continuous integration and Continuous delivery}

% Insbesondere Testautomatisierung wäre spannend, also die automatische Prüfung der Konstruktionsdaten.
The software can be extended with functions to merge and automatically test work results.
Thus, on the platform could perform an automatic check of the construction data files for test automation.

\subsubsection*{Release Management}

% Schließlich fände ich auch noch Release Management wichtig, also das Management der Zeitpunkte und Versionsstände, die dann in Produktion gehen.
One issue of the software when compared to GitHub is the lack of a release management. The software should be able to show at what time the different versions of a product have been created.