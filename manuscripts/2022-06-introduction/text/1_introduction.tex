\begin{abstract}
    Project management is facing new challenges as a result of the demand for increasingly individualized products in small batch sizes. The digitization of processes is an enabler for agile management. If product development is viewed separately from project management, many engineering artifacts are lost. This makes aspects such as scheduling, priority planning or the definition of new requirements difficult. A platform that combines product development with project management provides an overview of the entire project and gives the customer a transparent view of the progress. This article describes the development of such a platform and evaluates its benefits and drawbacks. For the development of the software, tools from web development were used to create a cross-platform user experience built on a client server architecture. The focus was on making the software lightweight and intuitive to use. The software was continuously tested during the development and showed potential to add value to the project management of products. It offers many possibilities for further development to expand the field of application. The software will be further developed in follow-up projects to make it ready for practical use and to extend its functions. There are already plans to extend the software in the direction of a simulation tool and virtual reality.

    ~\cite{Hackenberg2014}~\cite{Legat2014}~\cite{Teufl2015}
\end{abstract}

\section{Introduction}
    \label{sec:introduction}

    \subsection*{Context}
    %engineer to order und losgröße1 individualisierung
    Over time, customers demand more and more specific products at small batch sizes. Product cycles are becoming even shorter and the customer wants to get to his product in an agile and cost-efficient way. Engineer to order is a part of the production chain in which a product is produced and customized according to exact specifications. It is becoming increasingly important to involve the customer in the project management process in order to develop precise solutions for him. 
    % agilität
    An important part of this individual production consists of regular meetings. Through constant exchange, customer requirements can be discussed and specified. Agile product management makes it possible to adjust the product in the developing process if necessary. 
    % digitalisierung als enabler für die agilität 
    The digitization of development processes serves as an enabler for agility. It makes it possible to create a common basis for project management regardless of location. Customers can follow and evaluate the processes digitally. Through constant feedback, it is possible to react to customer wishes in order to reflect their interests. Agile product management is often supported by digital tools. With the help of spreadsheets and databases, the customer is involved in the development process. Project management and product development are not mapped together in such tools. Thus, depending on the product to be developed, it is difficult to give the customer a full overview of the progress.

    \subsection*{Problem}
    Requirements management is complicated by the fact that CAD data and documentation are kept separate. Because the customer does not have all of these necessary documents available in a clear manner, it is hard to follow the progress of the project. Requirement engineering task planning and scheduling are therefore difficult. By keeping the documentation in Excel lists or other documents, some engineering artifacts are lost. This lack of transparency in product development makes it difficult to implement agile product management. If the customer has no access to the development progress, there is no possibility of prioritizing times and requirements. He also cannot participate in design decisions and shape the product according to his wishes.

    \subsection*{Solution}
    To address these issues, we developed a product development platform. The platform offers user management, member management project management and provides a 3D view for the uploaded CAD models. On our platform, users can create different products. For each product, members, versions, issues, comments and milestones can then be added and edited. As in GitHub, different versions can be created. Each version of the product contain a 3D CAD model that can be viewed like in a 3D CAD software. Members can be registered for a product which possess different permissions. With the help of the customer member role, the customer can participate in agile project management and track the current status of product development. Issues, Comments and Milestones serve as project management tools. Milestones are filled with issues that can be open or closed. A milestone is considered complete when all associated issues are closed. Each issue contains a communication channel where the respective task can be evaluated. On the platform, the CAD model can be viewed in a 3D view and is connected to the integrated product management tools.

    \subsection*{Outline}
    This article is structured as follows:
    In Section~\ref{sec:differentiation} we discuss related work on the problem defined previously.
    In Section~\ref{sec:contribution} we present our original solution to the problem at hand.
    Then, in Section~\ref{sec:evaluation} we evaluate our solution with respect to different criteria.
    Finally, in Section~\ref{sec:conclusion} we summarize our learnings and provide an outlook on future work.
    
    \section{Related work}
    \label{sec:differentiation}
    % onshape, recherche bei google schoolar (problem als suchbegriff: requirement management, engineer to order)
    % wichtig um deutlich zu machen dass unser ansatz neu und wichtig ist
    % zentral: richtige suchbegriffe rumprobieren (manchmal werden dafür andere begriffe verwendet)
    % suchbegriffe notieren, systematik erklären können, wie man zu den quellen gekommen ist 
    % von 10000 ergebnisse erste 10 setien durchgescannt, anhand der kurzbeschreibung: relevant, vlt relevant, nicht relevant
    % relevante reingeschaut und dann erst entschieden 
    % wir form verwenden aktiv form überwiegend verwenden
    % papers: titel in google schoolar eingeben 

    % paper,github,onshape
    The search term requirement engineering results in 5 million entries on google schoolar. We scanned the first page of results and found many articles in the field of software development. On ScienceDirect, the search for requirement engineering returned over 900 000 entries. Here we found a broader range of use cases. Using the search term requirement management and product development we were able to find more specific articles on the topic on google schoolar. Project management is a huge subject, and it is difficult to specify specific issues. We have looked at some papers and consider the idea of combining project management with product development as new and important.

    A great representative and the model for this software is the platform GitHub. The difference is that GitHub is purely about the management of software. We want to create a broader application area with our app. 

    Another software that already exists on the market is Onshape. This software also combines product development with project management and offers a wide range of functions. We only discovered this software in the course of developing our own software. 
