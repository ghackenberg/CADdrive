\begin{abstract}
    Product development is facing new challenges as a result of the demand for increasingly complex and individualized products in small batch sizes and short time to markets. 
    By providing software tools, digitalization is an enabler for agile product development.
    Stakeholders such as customers, project managers, requirement engineers and product designers can develop the product together and react quickly to each other's demands.
    Today, each stakeholder uses his/her own platform to manage artifacts like product designs, requirements or schedules.
    The array of platforms complicates the information flow and frequently yields inaccuracies and inconsistencies.
    To solve these issues, we have developed an open source platform on which all relevant results can be managed.
    Our platform provides the means for storing product design revisions as well as design tasks and project schedules through an integrated data model.
    We complement this data model with a function model describing possible change operations on the data as well as a role-based permission model limiting the access to these operations.
    Finally, we propose an interface model through which the end users can access the design, task, and schedule data and execute the available operations.
\end{abstract}

\begin{IEEEkeywords}
    Product development, computer-aided design, requirements engineering, project management
\end{IEEEkeywords}

\section{Introduction}
    \label{sec:introduction}

    Today, customers across industries demand increasingly complex and individualized products with higher quality and shorter time to markets~\cite{Ahti2005}.
    At the same time, engineers need to deal with (partially) vague requirements specifications and customers expect flexibility with respect to change requests -- all due to uncertainties that cannot be resolved at the beginning of product development.
    Studies show that systematic requirements engineering with early stakeholder involvement and dedicated tool support represents half of the success of future product development~\cite{6226784}.
    To meet these challenges, agile methods have been proposed which increase the chances of delivering products with better quality in shorter time and with lower cost~\cite{ozkan2019agile}. 
    The majority of agile methods originate from the domain of software development, where traditional development methods frequently led to unsuccessful projects~\cite{HEIMICKE2021786}.
    In this domain, today established platforms exist -- such as GitHub\footnote{\url{https://www.github.com/}} -- which provide dedicated support for agile projects with high stakeholder involvement based on an integrated view on requirements, schedules, and deliverables.
    Unfortunately, due to the nature of the deliverables (mainly source code, binaries, and documentation), these platforms cannot be applied to product development directly~\cite{HEIMICKE2021786}. 
    During the development of physical products, typically various independent software tools are used for different activities, such as project management, requirements engineering, and product design~\cite{MarionTucker}.
    Because of the independence of these software tools, isolated artifacts are generated, which lack a common and integrated data model~\cite{houshmand2010collaborative}.
    The lack of a common and integrated data model, in turn, hinders the information flow between stakeholders and can cause inaccuracies and inconsistencies~\cite{Jorma2014}.
    These inaccuracies and inconsistencies finally lead to inefficiencies in product development, which cause unnecessary cost and project delays.
    This observation leads us to the following research questions.
    
    \subsection*{Research questions}
    How can we improve the information flow between customers, project managers, requirements engineers, and product designers in agile product development?
    How can we better integrate the available information about requirements, schedules, and deliverables in agile product development?
    How can we maybe translate the ideas from established software development platforms such as GitHub to the domain of product development?
    Which elements of these platforms can we reuse and which elements do we need to adapt to support product development activities directly?
    To answer these questions, we worked on a number of scientific contributions.

    \subsection*{Scientific contributions}
    We first provide an integrated \textbf{data model} of requirements, schedules, and deliverables, which supports management of CAD model revisions, linking of requirements to parts and assemblies of CAD model revisions, asynchronous discussion about and clarification of requirements, as well as prioritization and scheduling of requirements through milestones with fixed start and end dates.
    Then, we provide a \textbf{function model}, which explains the operations that stakeholders can perform on top of the data model during execution of agile product development projects.
    Furthermore, we provide a \textbf{permission model}, which limits the functions that project managers, engineers, and customers can execute based on their role in the project.
    Finally, we provide an \textbf{interface model}, which we derived from GitHub, but which required substantial changes to support the underlying data, function, and permission models properly.
    In this article we describe and discuss each of the contributions as outlined in the following.

    \subsection*{Article outline}
    In Section~\ref{sec:differentiation} we first discuss related work on requirements engineering and data integration in the context of agile product development.
    Then, in Section~\ref{sec:requirements} we summarize the requirements we believe an integration approach to project management, requirements engineering, and product design must fulfill.
    Thereafter, in Section~\ref{sec:contribution} we present the integrated data model, the function model, the permission model, and the interface model as explained in the previous section.
    Afterwards, in Section~\ref{sec:evaluation} we evaluate our models with respect to different criteria and unveil weaknesses, which need to be addressed eventually to be widely applicable.
    Finally, in Section~\ref{sec:conclusion} we summarize our learnings and provide an outlook on future work.
    
    \section{Related work}
    \label{sec:differentiation}
    We used Google Scholar\footnote{\url{https://scholar.google.com/}} to find related literature using the following keywords in various combinations:
    \begin{itemize}
        \item \textit{requirements} and (\textit{engineering} or \textit{management} or \textit{analysis} or \textit{specification})
        \item ((\textit{computer-aided} or \textit{mechanical} or \textit{industrial} or \textit{engineering} or \textit{product}) and \textit{design}) or (\textit{product} and \textit{development})
        \item \textit{agile} and (\textit{approach} or \textit{method} or \textit{methodology} or \textit{process} or (\textit{project} and \textit{management}))
        \item (\textit{product} and (\textit{lifecycle} or \textit{data}) and \textit{management}) or \textit{traceability}
    \end{itemize}
    Subsequently, we discuss the most relevant search results, which have been selected by title and abstract.
    
    \subsection*{Industry studies}
    Jassawalla and Sashittal~\cite{Jassawalla} study the importance of interdisciplinary collaboration in product development including customers and suppliers.
    Furthermore, the authors propose a collaboration framework based on principles such as shared responsibility as well as synergy between disciplines.
    Also, the authors stress success factors such as trust between project partners and support from top management.

    Sanchez and P{\'e}rez~\cite{sanchez2003flexibility} also show the importance of collaboration in interdisciplinary teams for product development.
    Therefore, the authors study practices in the automotive industry, while distinguishing between size of company and complexity of product.
    A main outcome is that strong collaboration within project teams results in better time-to-market and lower cost independent of company size and product complexity.
    
    B{\"u}y{\"u}K{\"o}zkan et al.~\cite{buyukozkan2004survey} highlight the importance of concurrent engineering and data integration in the agile manufacturing era.
    The authors summarize existing approaches, while distinguishing between collaboration, modeling and analysis, design synthesis and optimization, knowledge-based tools.
    Finally, they conclude with the observation, that the tools must be improved with respect to integration and agility.
    
    Durmuşoğlu and Barczak~\cite{DURMUSOGLU2011321} provide empirical evidence that software tools improve the effectiveness of product development projects.
    They use three measures to assess effectiveness of these projects: innovativeness, product quality, and market performance.
    The study suggests that the support for agile practices such as early stakeholder involvement should be improved in today's landscape.
    
    Marion and Fixson~\cite{marion_fixson_2019} have a deeper look into the tool landscape used during product development.
    The authors find that in general software tools help to share information and knowledge across team members, which applies specifically to virtual teams.
    However, the authors also suggest that general purpose tools might be easier to adopt by project teams than specialized solutions.
    
    \subsection*{Abstract frameworks}
    Kauppinen~\cite{Kauppinen2005} proposes a framework for integrating requirements engineering effectively into organizations.
    Besides a high-level process model the framework considers the cultural change as a major factor in improving organizational practices.
    While the author shows how requirements engineering practices can be improved effectively, the author does not deliver concrete data models and tools.
    
    Baxter et al.~\cite{BAXTER2008585} stress the importance of reusing knowledge generated in product development projects.
    Furthermore, the authors propose a framework for integrating requirements engineering with engineering design based on process, task, and product knowledge.
    However, the framework remains rather high-level and still does not provide information about a concrete technical realization of their ideas.

    Papinniemi et al.~\cite{Jorma2014} propose a framework for better integrating requirements management into product lifecycle management tools.
    Their approach discusses the relation between customer requirements and product designs as well as the challenges for adapting existing practices in industry.
    However, the authors do not explain how their ideas can be implemented practically into existing tool landscapes.

    Richter et al.~\cite{RICHTER2020271} propose a framework for visualizing the state and progress of product development projects.
    They summarize requirements of different stakeholders and propose a library of visualization techniques for different aspects of project progress.
    However, the authors do not provide ideas on how to store requirements, link requirements to product designs, and derive process information practically.

    \subsection*{Concrete tools for product development}
    Salo and Kakola~\cite{Ahti2005} evaluate the effectiveness of online collaboration platforms for requirements management in product development.
    Furthermore, the authors summarize the requirements for such platforms and propose a design, which is tested across several large-scale projects.
    However, the authors do not consider the integration of requirements with project schedules and design revisions.
    
    Liu et al.~\cite{liu2012scenario} provide a scenario based approach for the elicitation, decomposition, and formalization of engineering design requirements.
    Their approach tries to increase the quality of requirements specifications in the context of product development.
    While their approach provides better guidance on how to express requirements, their approach still lacks integration with project schedules and design revisions.
    
    Windisch et al.~\cite{WINDISCH2022550} propose a model-based approach for expressing requirements in mechatronic product development.
    Their approach is based on an activity-based view of design and verification tasks, which need to be carried out by product engineers.
    Through integration into agile project management tools, the requirements can be linked to project schedules, but the link to product design revisions is missing.
    
    \subsection*{Concrete tools for other domains}
    Goncalves de Branco et al.~\cite{6976693} propose a computer-aided software engineering (CASE) tool which helps to build accessible products.
    Their approach is based on an ontology, that allows one to describe technical implementations of accessibility requirements and, hence, enables traceability.
    Their approach is well suited for the software domain and accessibility, but cannot be used directly in product development.
    
    Belfadel et al.~\cite{belfadel2022requirements} propose a framework and tool for matching customer requirements with existing enterprise capabilities.
    Their approach is based on a multi-layer model of the enterprise architecture as well as patterns of customer requirements.
    The tool supports discovery of feasible capabilities and effectively fosters reuse, but its application to agile product development remains unclear.
    
    Mistler et al.~\cite{9447081} propose a modeling technique and select appropriate tooling for matching business requirements with organizational structures.
    To select appropriate tooling, the authors collect requirements that must be fulfilled in order for the tooling to be applicable.
    Again, the approach helps in arranging and combining enterprise capabilities, but its application to product development remains unclear.

    \subsection*{Literature summary}
    Most papers we have found stress the importance of requirements engineering and data integration for successful product development.
    Some papers provide abstract frameworks for integrating requirements engineering deeper into the product development process.
    Only few articles describe concrete data models and tools to deliver  integrated support for customers, project managers, requirements engineers, and product designers.
    In the software domain, this data model and tool-based integration across disciplines has been established more deeply.
    For product development, however, a complete integration of the disciplines is still missing.