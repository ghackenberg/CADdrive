\begin{abstract}
    Product development is facing new challenges as a result of the demand for increasingly complex and individualized products in small batch sizes and short time to markets. 
    By providing software tools, digitalization is an enabler for agile product development.
    Stakeholders such as customers, project managers, requirement engineers and product designers can develop the product together and react quickly to each other's demands.
    Today, each stakeholder uses his/her own platform to manage artifacts like product designs, requirements or schedules.
    The array of platforms complicates the information flow and frequently yields inaccuracies and inconsistencies.
    To solve these issues, we have developed an open source platform on which all relevant results can be managed.
    Our platform provides the means for storing product design revisions as well as design tasks and project schedules through an integrated data model.
    We complement this data model with a function model describing possible change operations on the data as well as a role-based permission model limiting the access to these operations.
    Finally, we propose an interface model through which the end users can access the design, task, and schedule data and execute the available operations.
\end{abstract}

\begin{IEEEkeywords}
    Product development, computer-aided design, requirements engineering, project management
\end{IEEEkeywords}

\section{Introduction}
    \label{sec:introduction}

    Today, customers across industries demand increasingly complex and individualized products with higher quality and shorter time to markets~\cite{Ahti2005}.
    At the same time, engineers need to deal with (partially) vague requirements specifications and customers expect flexibility with respect to change requests -- all due to uncertainties that cannot be resolved at the beginning of product development.
    Studies show that systematic requirements engineering with early stakeholder involvement and dedicated tool support represents half of the success of future product development~\cite{6226784}.
    To meet these challenges, agile methods have been proposed which increase the chances of delivering products with better quality in shorter time and with lower cost~\cite{ozkan2019agile}. 
    The majority of agile methods originate from the domain of software development, where traditional development methods frequently led to unsuccessful projects~\cite{HEIMICKE2021786}.
    In this domain, today established platforms exist -- such as GitHub\footnote{https://www.github.com/} -- which provide dedicated support for agile projects with high stakeholder involvement based on an integrated view on requirements, schedules, and deliverables.
    Unfortunately, due to the nature of the deliverables (mainly source code, binaries, and documentation), these platforms cannot be applied to product development directly~\cite{HEIMICKE2021786}. 
    During the development of physical products, typically various independent software tools are used for different activities, such as project management, requirements engineering, and product design~\cite{MarionTucker}.
    Because of the independence of these software tools, isolated artifacts are generated, which lack a common and integrated data model~\cite{houshmand2010collaborative}.
    The lack of a common and integrated data model, in turn, hinders the information flow between stakeholders and can cause inaccuracies and inconsistencies~\cite{Jorma2014}.
    These inaccuracies and inconsistencies finally lead to inefficiencies in product development, which cause unnecessary cost and project delays.
    This observation leads us to the following research questions.
    
    \subsection*{Research questions}
    How can we improve the information flow between project managers, requirements engineers, and product designers in agile product development?
    How can we better integrate the available information about requirements, schedules, and deliverables in agile product development?
    How can we maybe translate the ideas from established software development platforms such as GitHub to the domain of product development?
    Which elements of these platforms can we reuse and which elements do we need to adapt to support product development activities directly?
    To answer these questions, we worked on a number of scientific contributions.

    \subsection*{Scientific contributions}
    We first provide an integrated \textbf{data model} of requirements, schedules, and deliverables, which supports management of CAD model revisions, linking of requirements to parts and assemblies of CAD model revisions, asynchronous discussion about and clarification of requirements, as well as prioritization and scheduling of requirements through milestones with fixed start and end dates.
    Then, we provide a \textbf{function model}, which explains the operations that stakeholders can perform on top of the data model during execution of agile product development projects.
    Furthermore, we provide a \textbf{permission model}, which limits the functions that project managers, engineers, and customers can execute based on their role in the project.
    Finally, we provide an \textbf{interface model}, which we derived from GitHub, but which required substantial changes to support the underlying data, function, and permission models properly.
    In this article we describe and discuss each of the contributions as outlined in the following.

    \subsection*{Article outline}
    In Section~\ref{sec:differentiation} we first discuss related work on requirements engineering and data integration in agile product development.
    Then, in Section~\ref{sec:contribution} we present our integrated data model, the function model, the permission model, and the interface model as explained in the previous section.
    Thereafter, in Section~\ref{sec:evaluation} we evaluate our models with respect to different criteria and unveil weaknesses, which need to be addressed eventually to be widely applicable.
    Finally, in Section~\ref{sec:conclusion} we summarize our learnings and provide an outlook on future work.
    
    \section{Related work}
    \label{sec:differentiation}
    In the following, we look at related work on requirements engineering and data integration for agile product development. Therefore, first we summarize related approaches, which have been published by the scientific community in recent years. Then, we summarize related commercial approaches, which are available on the market today. Finally, we draw our conclusions on the state of the art.

    \subsection*{Scientific approaches}
    \label{sec:scientific}
    Google Scholar was used to search for appropriate literature using the following search terms in various combinations: Requirements Engineering, requirements management, computer aided design, product development, mechanical engineering, mechanical design, industrial design, traceability, integration, product lifecycle management, software tools, agile project management, new product development, collaboration, collaborative information technology. For each combination searched, the first 30 entries were looked through, and the most relevant articles were used as state-of-the-art references. 
    
    Most of the articles focus on requirement management and describe its theoretical aspects. These articles also point out the importance of well-executed requirement management and the challenges that occur in the process. It is mentioned again and again that requirement engineering is a major part of success.
    We found an article that divides requirement management into subcategories and phases, describing each aspect of developing new products in a high-tech company~\cite{Ahti2005}.
    In product development, it is important to be able to reuse the knowledge generated. One article described a framework for combining requirements management with engineering design and making it reusable. The resulting design reuse system consists of three key elements: process knowledge, task knowledge and product knowledge~\cite{BAXTER2008585}.
    An article provides a scenario based approach for the identification, elaboration and specification of engineering design requirements using a three-phase model~\cite{liu2012scenario}.
    Another article provides a literature review to show how requirements engineering can be further enhanced with the help of guidelines for requirement management process improvement~\cite{Kauppinen2005}.
    Computer aided software engineering tools help to support the development process. In one article such a CASE tool was developed as an Eclipse plugin which enables traceability and accessibility from the planning to the coding phase. This work focuses on the development of software products~\cite{6976693}.
    A survey evaluated the use of individual practices and cooperation activities for new product development depending on firm size and complexity of the products~\cite{sanchez2003flexibility}.
    An article shows the importance for requirements engineering to use visualization tools to help stakeholders catch data. They describe different elements for visualization and want to implement a software tool based on their results later~\cite{RICHTER2020271}.
    An article points out what methods and tools are important for combining requirements engineering with project management and how important it is to see the two fields as an overall concept. The challenges of requirements engineering and project management in different contexts are highlighted. A framework is developed that combines requirements management and project management to increase the quality and effectiveness of multidomain development processes. The integration of CAD data is not covered~\cite{Jorma2014}.
    Another contribution is the implementation of a tool for model based requirement engineering for agile product development. The concept of the tool is to connect the product requirement with the development task and the corresponding test case. Based on this a dashboard was built which received positive feedback and will be further developed~\cite{WINDISCH2022550}.
    One paper deals with concepts for concurrent new product development to implement agile manufacturing. The most important tools and concepts are presented and discussed in how far they enable agile manufacturing~\cite{buyukozkan2004survey}.
    A tool was built using web development tools such as Angular.js and Node.js to support product development. The tool supports the business, operational and technical aspects for service-oriented software~\cite{belfadel2022requirements}.
    Another paper compared different software tools for enhanced demand compliant design based on different criteria. The analysis shows how important it is to use the right tool and that most tools focus only on the own discipline and rather a universal and transdisciplinary approach should be chosen~\cite{9447081}.
    We have found a study that provides empirical evidence that tools for new product development have a positive impact on efficiency. They suggest three measures of effectiveness: innovativeness, new product quality and market performance. most tools do not cover all three aspects, as they are often very specific to one use case~\cite{DURMUSOGLU2011321}.
    An article presents a framework to improve cross-functional collaboration in a new product development process. They emphasize that innovative and structural mechanisms increase the level of integration and effective management~\cite{Jassawalla}.
    In one article, the importance of requirement engineering is discussed and attention is drawn to the fact that most studies and research work that is available has many strategies but no real implementation~\cite{kumar2022requirements}.
    A paper compared two projects were compared with each other to evaluate the impact of collaborative information technology tools. Project A used software for communication, tasks and data storage. Project B divided these parts among three tools and experienced significant delays and cost overruns. The manager of project A noted the benefit of a single place for communication and product iteration~\cite{marion_fixson_2019}.

    \subsection*{Commercial approaches}
    \label{sec:commercial}

    \subsubsection*{GitHub}
    GitHub is a service for the management of agile software projects. Repositories are created and managed here. Multiple project versions can be uploaded to the repository and requirements can be collected, divided into milestones and discussed in the issues section. These functionalities solve some problems defined above. However, GitHub is only designed for software products and does not support CAD files. Thus, solution engineering of mechanical systems cannot be included. We find GitHub to be an excellent platform for software development and in the course of this article we want to create a platform that offers similar tools, but is suitable for 3D CAD models.

    \subsubsection*{PTC Onshape\footnote{https://www.onshape.com/}}
    The tool onshape is a premium software, from the company PTC, which combines product development, the CAD, data management, collaboration and real-time analysis. The tool offers a huge range of features. It provides a secure cloud workspace for all project stakeholders and the development team can work together on the design of the product. The tool enables multiple parallel design iterations and real-time collaborative work, and offers many more features. 
    Unfortunately, the software also has disadvantages compared to lightweight open source software due to the huge scope of functions and the binding to one company. If the customer is involved in an agile project development process, he has a high entry hurdle when learning the program.  In order to keep the tool consistent with real project progress, it must be constantly updated. Due to the high amount of data in the tool, the stakeholders and especially the customer can lose the overview. For the customer, it is easier to see exports of product versions than the entire system when making decisions, and the company does not have to disclose design details. The deep integration of Onshape into the world of PTC forces you to be caught in it. 
    Here we want to go the approach of an open source software that works with 3D CAD exports. Furthermore, the software should be intuitive to use and easy to learn. Because the entire solution engineering process does not take place in our tool, but only exports are uploaded, there is also freedom with which data the platform is fed and what the customer gets to see. The customer should be able to participate in the product development on the basis of the product versions provided without having to see complex design details.

    \subsection*{Conclusion}
    Most of the papers we have found through research highlight the area of requirements engineering in particular or provide concepts for new product development. They provide many theoretical concepts to point out the importance, challenges and methods for improvement. 
    Only a few articles describe a concrete implementation, but these only cover requirements engineering and project management or a combination of these two disciplines. None of these solutions included the possibility to additionally integrate CAD data to cover the entire product development process. The two commercial solutions GitHub and Onshape are excellent tools in their own domain but in our opinion are not suitable to support the agile product development process with close customer collaboration because of various reasons. 
    In the case of GitHub, it's because it doesn't support design drawings. 
    Oneshape software is an expert tool that delivers a huge feature set and discloses confidential enineering data. Customer and other stakeholders can easily lose track by access all the design details of the product. After our research we came to the conclusion that there is a need for a lightweight software tool that covers requirement engineering, project management and offers the possibility to integrate 3D CAD models. This field of research is new and important and the market is by far not yet saturated with tools that provide support for product development.
