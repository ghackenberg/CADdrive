\begin{abstract}
    Product development is facing new challenges as a result of the demand for increasingly individualized products in small batch sizes and short time to markets. 
    By providing software tools, digitalization is an enabler for agile product development. Stakeholders such as the customer, the project manager, the requirement manager and the solution manager can develop the product together and react quickly to each other's demands.
    The individual aspects of product development are often considered separately. Each stakeholder uses its own platform to generate outputs. As a result, the connection between the components of product development is missing. Without a common context between the different departments, valuable information and work results are lost.
    This complicates aspects such as scheduling, priority planning or the definition of new requirements as there is no common basis of information.
    To solve these issues, we have developed an open source platform on which all relevant results can be managed.
    Our platform combines project management, requirements engineering and solution engineering to offer a transparent insight into the progress of product development. 
    For the development of the software, tools from web development were used to create a cross-platform user experience built on a client server architecture. The focus was on making the software lightweight and intuitive to use. 
    The software was continuously tested during the development process and showed potential to add value to the project management of products. Nevertheless, important adjustments still need to be made to create a solid user experience.
    This article describes the development of such a platform and evaluates its benefits, drawbacks and what lessons we have learned. 
    In the Future the software will be further developed in follow-up projects to make it ready for practical use and to extend its functions. 
    There are already plans to extend the software in the direction of a simulation tool and virtual reality.
\end{abstract}

\section{Introduction}
    \label{sec:introduction}
    \subsection*{Context}
    Every engineering design process starts with defining the needs to be covered in the resulting product. In order to successfully develop the product, it is important to capture the customer's needs precisely and effectively~\cite{liu2012scenario}.
    The trend is for products to become increasingly complex and shorter time to markets.
    To meet today's challenges, companies must shorten the cycle time of new products, reduce the resources used, increase the quality of the product. To meet this business goals, companies need to collect, analyze and apply requirements extensively and effectively~\cite{Ahti2005}. Globalization and digitalization are increasing the demand for individual products. To meet these challenges, agile methods are introduced into the development process~\cite{HEIMICKE2021786}. Requirement engineering plays an important role as a discipline and holds many challenges. Clear requirements and a focus on the end user with management support represent half of the success of future product development~\cite{6226784}.
    Through proper requirement management, a product with higher quality can be developed in less time by meeting the right requirements because fewer iterations are required. The customer satisfaction is increased and through a clean documentation it is possible to reuse parts of the process for other products~\cite{BAXTER2008585}.
    By collecting the information and by possible changes of the customer requirements, a lot of data comes up, which makes it difficult to keep an overview of the product development process. It is difficult to manage the large amount of data because of the cognitive capacity of humans. Project management tools help to get an overview of the project progress with visualizations in the form of dashboards~\cite{RICHTER2020271}.
    Using agile project management tools leads to more speed, efficiency and influences the quality of the product. The quality of the final product is mostly related to the project management.~\cite{ozkan2019agile}. 
    Collaborative Information Technology (CIT) tools have a significant impact on the development of new products. Such tools are used for example for project management, digital design, project communication, data storage, simulation tools. They enhance the process of knowledge generation through the rapid distribution of ideas, comments, and design changes. Most of these CIT tools are specialized for a specific task and very few tools offer a multifunctional solution.~\cite{MarionTucker}.
    
    
    
    \subsection*{Problem}
    The majority of agile approaches come from software development and are not easily transferable to product development, where a very high level of complexity dominates~\cite{HEIMICKE2021786}
    The development of complex and sustainable products produces also many challenges for requirement management. Requirement management is an effort where data from different stakeholders merge. 
    Changing requirements in one domain also affects other domains. Many tools already exist to support requirement management in the different domains. Due to the distribution of tools, stakeholders often have difficulty defining priorities during the product development phase because they do not have an overview of the big picture. Commercial tools can mostly be used to support requirement engineering and the project management process. If these tools are separate applications and follow different processes, merging the data becomes a major challenge. The division into different systems leads to differences in terminology and concepts, which causes communication problems~\cite{Jorma2014}.
    The relationship between effort and benefit in a software tool must also be considered. It must be taken into account what maximum effort the stakeholders are willing to invest to use the software. In a tool that is not comprehensibly structured and difficult to learn, not much effort will be invested, and no added value can be generated~\cite{RICHTER2020271}.
    Such a tool must take into account the dynamics of the requirements in the development process and provide support for collaborative and interdisciplinary cooperation. It should be possible to divide the engineering activities into sprints~\cite{liu2012scenario}.

    

    Even with a tool that covers requirements engineering and project management, many work results are lost because not all aspects of product development are included. Because CAD data and documentation are often considered separately from management, the context is lost. 
    Requirement engineering task planning and scheduling are therefore difficult.
    This lack of transparency in product development makes it hard to implement agile product management. If the customer has no access to the development progress, there is no possibility of prioritizing times and requirements. 

    \subsection*{Solution approach}
    Since tools to support product development do not cover all essential aspects such as project management, requirement engineering and solution engineering, we want to describe the development of such software and evaluate it in the course of this article. The software considers product development as an overall concept. This gives the stakeholders a transparent and traceable overview of the project progress and enables agile product development.
    The platform offers user management, member management project management and provides a 3D view for the uploaded CAD models. On our platform, users can create different products. For each product, members, versions, issues, comments and milestones can then be added and edited. As in GitHub, different versions can be created. Each version of the product contain a 3D CAD model that can be viewed like in a 3D CAD software. Members can be registered for a product which possess different permissions. With the help of the customer member role, the customer can participate in agile project management and track the current status of product development. Issues, Comments and Milestones serve as project management tools. Milestones are filled with issues that can be open or closed. A milestone is considered complete when all associated issues are closed. Each issue contains a communication channel where the respective task can be evaluated. On the platform, the CAD model can be viewed in a 3D view and is connected to the integrated product management tools.

    \subsection*{Outline}
    This article is structured as follows:
    In Section~\ref{sec:differentiation} we discuss related work on the problem defined previously.
    In Section~\ref{sec:contribution} we present our original solution to the problem at hand.
    Then, in Section~\ref{sec:evaluation} we evaluate our solution with respect to different criteria.
    Finally, in Section~\ref{sec:conclusion} we summarize our learnings and provide an outlook on future work.
    
    \section{Related work}
    \label{sec:differentiation}

    \subsection*{Scientific approach}
    Google Scholar was used to search for appropriate literature using the following search terms in various combinations: Requirements Engineering, requirements management, computer aided design, product development, mechanical engineering, mechanical design, industrial design, traceability, integration, product lifecycle management, software tools, agile project management, new product development, collaborative information technology. For each combination searched, the first 30 entries were looked through, and the most relevant articles were used as state-of-the-art references. Most of the articles focus on requirement management and describe its theoretical aspects. These articles also point out the importance of well-executed requirement management and the challenges that occur in the process. It is mentioned again and again that requirement engineering is a major part of success.
    We found articles that divide requirement management into subcategories and phases, describing each aspect of developing new products in a high-tech company~\cite{Ahti2005}.
    In product development, it is important to be able to reuse the knowledge generated. One article described a framework for combining requirements management with engineering design and making it reusable. The resulting design reuse system consists of three key elements: process knowledge, task knowledge and product knowledge~\cite{BAXTER2008585}.
    An article provides a scenario based approach for the identification, elaboration and specification of engineering design requirements using a three-phase model~\cite{liu2012scenario}.
    Another article provides a literature review to show how requirements engineering can be further enhanced with the help of guidelines for requirement management process improvement~\cite{Kauppinen2005}.
    Computer aided software engineering tools help to support the development process. In one article such a CASE tool was developed as an Eclipse plugin which enables traceability and accessibility from the planning to the coding phase. This work focuses on the development of software products~\cite{6976693}.
    An article shows the importance for requirements engineering to use visualization tools to help stakeholders catch data. They describe different elements for visualization and want to implement a software tool based on their results later~\cite{RICHTER2020271}.
    An article points out what methods and tools are important for combining requirements engineering with project management and how important it is to see the two fields as an overall concept. The challenges of requirements engineering and project management in different contexts are highlighted. A framework is developed that combines requirements management and project management to increase the quality and effectiveness of multidomain development processes~\cite{Jorma2014}.
    Another contribution is the implementation of a tool for model based requirement engineering for agile product development. The concept of the tool is to connect the product requirement with the development task and the corresponding test case. Based on this a dashboard was built which received positive feedback and will be further developed~\cite{WINDISCH2022550}.
    In addition, a tool was built using web development tools such as Angular.js and Node.js to support product development. The tool supports the business, operational and technical aspects for service-oriented software~\cite{belfadel2022requirements}.
    Another paper compared different software tools for enhanced demand compliant design based on different criteria. The analysis shows how important it is to use the right tool and that most tools focus only on the own discipline and rather a universal and transdisciplinary approach should be chosen~\cite{9447081}.
    An article presents a framework to improve cross-functional collaboration in a new product development process. They emphasize that innovative and structural mechanisms increase the level of integration and effective management~\cite{Jassawalla}.
    In one article, the importance of requirement engineering is discussed and attention is drawn to the fact that most studies and research work that is available has many strategies but no real implementation~\cite{kumar2022requirements}.
    A paper explored how collaborative information technology tools can help in the development of new products. Two projects were compared with each other. Project A used software for communication, tasks and data storage. Project B divided these parts among three tools and experienced significant delays and cost overruns. The manager of project a noted the benefit of a single place for communication and product iteration~\cite{marion_fixson_2019}.

    \subsection*{Commercial approach}
    \subsubsection*{GitHub}
    It is a service for version management of software projects. Repositories are uploaded and managed here. Multiple versions can be uploaded from the repository and tasks can be distributed, divided into milestones and discussed in the issues section. These functionalities solve some problems defined above. However, GitHub is only designed for software products and does not support CAD files. Thus, solution engineering of mechanical systems cannot be included. We find GitHub to be an excellent platform for software development and in the course of this article we want to create a platform that offers similar tools, but is suitable for 3D CAD models~\footnote{https://github.com/}.

    \subsubsection*{Onshape}
    The tool onshape is a premium software, from the company PTC, which combines product development, the CAD, data management, collaboration and real-time analysis. The tool offers a huge range of features. It provides a secure cloud workspace for all project stakeholders and the development team can work together on the design of the product. The tool enables multiple parallel design iterations and real-time collaborative work, and offers many more features. 
    Unfortunately, the software also has disadvantages compared to lightweight open source software due to the huge scope of functions and the binding to one company. If the customer is involved in an agile project development process, he has a high entry hurdle when learning the program. The customer can get lost in the tool and the actual aspects of the product to be developed can be lost in the process. In order to keep the tool consistent with real project progress, it must be constantly updated. Due to the high amount of data in the tool, the stakeholders and especially the customer can lose the overview. For the customer, it is easier to see exports of product versions than the entire system when making decisions, and the company does not have to disclose design details. The deep integration of Onshape into the world of PTC forces you to be caught in it. 
    Here we want to go the approach of an open source software that works with 3D CAD exports. Furthermore, the software should be intuitive to use and easy to learn. Because the entire development process does not take place in our tool, but only exports are uploaded, there is also freedom with which data the platform is fed and what the customer gets to see. The customer should be able to participate in the product development on the basis of the product versions provided without having to see complex design details~\footnote{https://www.onshape.com/}.

    \subsubsection*{Conclusion}
    Most of the papers we have found through research highlight the area of requirements engineering from the product development process in particular. They provide many theoretical concepts to point out the importance, challenges and methods for improvement. Only a few articles describe a concrete implementation, but these only cover requirements engineering and not the entire product development process. The two commercial solutions GitHub and Onshape are excellent tools in their domain but in our opinion are not suitable to support the agile product development process with close customer collaboration. In the case of GitHub, it's because it doesn't support design drawings and oneshape software is an expert tool that delivers a huge feature set, which can make the customer lose track by access all the design details of the product. After our research we came to the conclusion that our field of research is new and important and the market is by far not yet saturated with tools that provide lightweight support for product development.
