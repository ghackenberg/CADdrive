\section{Critical evaluation}
\label{sec:evaluation}
To solve the issues defined at the beginning of the article, we have developed the software ProductBoard. This software combines agile project management with product development. It was developed as a team of two people and the development time is currently one year. Due to the short development time, the software has of course not yet reached its final state. Nevertheless, it already covers a wide range of required issues. The development was done with state-of-the-art technology. The clean software architecture makes it easy to extend and maintain the software. As a single user, the software could already be tested at this stage. The user interface is responsive, and the software is intuitive to use. At this point, the app is still designed for a single user. There are still no functionalities built in that several PCs can access the data system simultaneously. This requires a revision in the backend to be able to react live to changes and display them again in the frontend. Another issue is that the data is consistent across all clients. One solution is a server client architecture with a web socket. For an adaptation to the customer, the permission system must also be adapted individually. Furthermore, the software has to be tested for stability. The current state of the project shows that it is possible and realistic to generate a benefit for agile product management and product development. In the next steps, the app can be made ready to enable communication between multiple clients. For this purpose, the current state can be continued and further developed without a complete revision. 

\section{Conclusion}
\label{sec:conclusion}

\subsection*{Summary}
To create a common basis for product development and product management, the software ProductBoard was developed. The advantage of this combination is that no engineering artifacts are lost. By bundling on a single app, a better overview of the process and more transparency is created. The customer can participate in agile product management and at the same time has the current status of all versions of the products. For this purpose, users can create and manage products on the platform. These products have versions, issues, comments and milestones. Members can also be added to the products, which have different permissions. There are three roles: manger, engineer and customer. The software offers versioning similar to GitHub. Based on the previous version and the current version, the history can be displayed graphically. Via issues, it is possible to assign tasks to one or more assignees. These tasks can be discussed in the associated conversation channel. After completion of a task, it can be marked as closed. Milestones are used to create an overview of the issues. Through the milestones the progress of the project can be tracked well. For the development of the software we decided to use tools from web development. This enables a cross-operating system user experience and can be well-built on a client server architecture. The three main components are the backend, the frontend and the database. In the center is the data model and the API for read and write operations to the data model. The frontend serves as a generic layer for triggering  operations and as a graphical representation of data. To enable operation on company and customer level a permission model was implemented. This can be customized according to the customer's needs by adjusting the customer's permissions individually. The functionality from the user's point of view was tested each time a new functionality was added. The tests have shown us that it is already possible to manage products on a PC using the software. In order to be ready for a wider range of users, the app still needs to be adapted. Due to the clear separation of frontend and backend, it is possible to extend the app in this direction without major reconstruction measures. 

\subsection*{Outlook}
According to our tests, the app has the potential to generate real added value for customers and companies. In order for the app to run on multiple devices at the same time and access a central backend, adjustments still need to be made. Furthermore, it has to be ensured that the software runs stable independent of all user inputs. This will be checked in a test phase. The architecture of the software was designed in such a way that other files can be uploaded in the future instead of CAD data. We plan to expand the scope of the app in the future. One idea is for example the management of electrical schematics. Like 3D CAD files, these can be designed and evaluated according to customer requirements. The software can also evolve in the direction of a simulation platform. It would be possible to upload entire simulated processes and support them with project management tools. For example, mechanical or electrical systems could be simulated directly in the View. For this, only the 3D view would have to be replaced by the corresponding functionality. Furthermore, we see potential in VR applications. With the help of VR, communication between companies and customers could be greatly improved during the project phase. The customer would have the possibility to view the 3D CAD model with all its versions in virtual space. This would allow the customer to move the model freely in space and thus capture even more engineering artifacts. These functions are already partially implemented in the software and will be extended in the future. A follow-up project is already in the planning phase and will be launched in the next few months. The follow-up project will complement this software with VR functions to enable an evaluation of the product development process in virtual space. 