\documentclass[onecolumn]{PDS}
%%%% Packages
\usepackage{graphicx}
\usepackage{multicol,multirow}
\usepackage[fleqn]{amsmath}
\usepackage{amssymb,amsfonts}
\usepackage{mathrsfs}
\usepackage{amsthm}
\usepackage[figuresright]{rotating}
\usepackage{appendix}
\usepackage[authoryear]{natbib}
\usepackage{ifpdf}
%\usepackage{newtxtext}
%\usepackage{newtxmath}
\usepackage[T1]{fontenc}
\usepackage{textcomp}
\usepackage{xcolor}
\usepackage[colorlinks,allcolors=dscolor,bookmarks=false]{hyperref}

\jdoi{https://doi.org/10.1017/pds.2025.xxx}

\usepackage{boites}

\begin{document}

\articletype{}% Leave empty

\title[]{How to use the PDS \LaTeX\ class}


\maketitle

\hrule{}
\vspace{35mm}
Leave empty for metadata
\vspace{35mm}
\hrule{}


\section{Introduction}\label{sec1}

This latex class file is available for the authors to prepare
the manuscript for PDS Journal. It is assumed that
the authors are familiar with either plain \TeX, \LaTeX,\
\AmS-\TeX\ or a standard latex set-up, hence, only the
essential points are described in this document. To get more
details please go through the \textit{\LaTeX\ User's Guide} or
\textit{The not so short introduction to \LaTeXe} (which is available online).
The \verb|PDS.cls| is similar as the \texttt{article.cls} of
\LaTeX, with only few additional changes in the preamble portion.

\section{Installation}
The \texttt{PDS.cls} has to be copied into a directory
where tex looks for input files. The other files has to
keep as a reference during the preparation of your
manuscripts. Please use pre-defined commands from
for title, authors, address, abstract, keywords, body etc. as shown in Box 1.

\section{How to start using PDS.cls}

Before you type anything that actually appears in the paper
you need to include a\break \verb|\documentclass{PDS}|
command at the very beginning and then, the two commands
that have to be part of any latex document,
\verb|\begin{document}| at the start and the
\verb|\end{document}| at the end of your paper. The main
structure of your document should be as follows:

\vskip1pc
\noindent Box 1: Structure of a document.

\begin{breakbox}
\begin{verbatim}
\documentclass{PDS} %%% For double column layout.

%%% In case if you want the article in single column, then please use
%%% the option "onecolumn" in the optional of document class as shown below
%%% Also, if you want to submit your article in 11pt size, then please use
%%% the option xipt in the document class as shown below.

%%% \documentclass[onecolumn,xipt]{PDS}

\begin{document}

\title{How to use the PDS \LaTeX\ class}

\author[1]{First author}
\author[2]{Second author}
\author*[3]{Third author - Corres author}

\address[1]{First author address,}
\address[2]{Second author address,}
\address[3]{Third author address}

\corresemail{xxxx@xxxx.xxx.xx}

\abstract{abstract text }

\keywords{keyword entry 1, keyword entry 2, keyword entry 3}

\maketitle

\section{....}
...
\subsection{....}
....
\end{document}
\end{verbatim}
\end{breakbox}

\vspace*{2pc}

\section{Preamble part}

All the options in \texttt{article.cls} are
available with this class file, by default it
will produce all elements single spaced throughout the
document.

By default, PDS class file produce numbered bibliography.

\subsection{Paper Title}

The paper title is declared like: \verb|\title{...}| in the standard LATEX manner.
Line breaks \verb|\\| may be used to equalize the length of the title lines.

\subsection{Author Names}

The name and associated information is declared with the
\verb|\author| command. \verb|\author| behaves slightly differently
depending on the document mode. For more details about author information see Box 1.

\subsection{Abstract \& Keywords}

The abstract is generally the first part of a paper. The abstract text is placed within the abstract
environment.

\newpage
Keywords should be inserted immediately after the abstract text with grouping as shown below.

\begin{verbatim}
\abstract{
Abstract text here
}

\keywords{Keyword text here}
\end{verbatim}

\section{Body part}

\subsection{Sections}
\label{sec3.6}

The coding for section is \verb|\section{text}|. This will generate section number automatically.
Use the starred form (\verb|\section*{text}|) of the command to suppress the automatic numbering.
If you want to make cross references to the section levels use the \verb|\label| and \verb|\ref| command.
You can have sections up to five levels.

The sectioning commands are \verb|\section|, \verb|\subsection|, \verb|\subsubsection|,
\verb|\paragraph| and \verb|\subparagraph|.


\subsection{Figures and tables}\label{sec3.9}

Use the default \LaTeX\ coding for figures and tables.
Figure and table environments should be inserted after the end of the paragraph,
nearest to the citation.


\noindent The coding for figure is:

\begin{verbatim}
\begin{figure}[!h]
\centering{\includegraphics{sample.eps}}
\caption{Insert figure caption\label{fig1}}
\end{figure}
\end{verbatim}

\noindent The coding for table is:

\begin{verbatim}
\begin{table}[!t]
\centering
\caption{Insert table caption her\label{tab1}}
\begin{tabular*}{\textwidth}{@{\extracolsep{\fill}}lllll@{}}
\Toprule
Column head 1 & Column head 2 & Column head 3 &
Column head 4 & Column head 5\\
\midrule
Table body & Table body & Table body  & Table body  & Table body  \\
Table body & Table body & Table body  & Table body  & Table body  \\
Table body & Table body & Table body  & Table body  & Table body  \\
Table body & Table body & Table body  & Table body  & Table body  \\
Table body & Table body & Table body  & Table body  & Table body  \\
\botrule
\end{tabular*}
\end{table}\end{verbatim}

As always with \LaTeX, the \verb|\label| must be after the
\verb|\caption|, and inside the figure or table environment. The reference for
figures and tables inside text can be made using the \verb|\ref{key}| command.


\subsection{Equations}\label{sec3.10}

Equations are used in the same way as described in the \LaTeX\ manual.
Equations are numbered consecutively, with equation numbers
in parentheses flush right.\noindent

\newpage
\noindent For example, if you type
\begin{verbatim}
\begin{equation}\label{eq1}
\int^{r_2}_0 F(r,\varphi){\rm d}r\,{\rm d}\varphi = [\sigma r_2/(2\mu_0)]
\int^{\infty}_0\exp(-\lambda|z_j-z_i|)\lambda^{-1}J_1 (\lambda r_2)J_0
(\lambda r_i\,\lambda {\rm d}\lambda)
\end{equation}
\end{verbatim}
then you will get the following output:
\begin{equation}\label{eq1}
\int^{r_2}_0 F(r,\varphi){\rm d}r\,{\rm d}\varphi = [\sigma r_2/(2\mu_0)]\int^{\infty}_0
\exp(-\lambda|z_j-z_i|)\lambda^{-1}J_1 (\lambda r_2)J_0 (\lambda r_i\,\lambda {\rm d}\lambda)
\end{equation}
\AmS-\LaTeX{} has several environments that
make it easier to typeset complicated multiline displayed equations. These are explained in the
\AmS-\LaTeX{} User Guide. A \verb|subequation| environment is available to create equations with
sub-numbering of the equation counter. It takes one (optional)
argument to specify the way that the sub-counter should appear.


\subsection{Quotes and displayed text}
\label{sec3.11}

Quotes are indented from the left and right margins. There are various types of quotes,
short quote, long quote and display poetry.\vskip1pc

\noindent The coding for short quote is \verb|\begin{quote}...\end{quote}|.
\begin{quote}
   This is a short quotation.  It consists of a
   single paragraph of text.  See how it is formatted.
\end{quote}

\noindent The coding for long quote is \verb|\begin{quotation}...\end{quotation}|.

\begin{quotation}
   This is a longer quotation.  It consists of two
   paragraphs of text, neither of which are
   particularly interesting.

   This is the second paragraph of the quotation.  It
   is just as dull as the first paragraph.
\end{quotation}

\subsection{Listings}
\label{sec3.12}

Another frequently displayed structure is a list. There are various types of list
numbered, itemized and bulleted list.

\noindent The coding for bulleted list are as follows:

\begin{verbatim}
\begin{itemize}
\item Bulleted list 1
\item Bulleted list 2
\item Bulleted list 3
\end{itemize}
\end{verbatim}

\noindent The coding for numbered list are as follows:

\begin{verbatim}
\begin{enumerate}
\item Numbered list 1
\item Numbered list 2
\item Numbered list 3
\end{enumerate}
\end{verbatim}

\noindent The coding for description list are as follows:

\begin{verbatim}
\begin{description}
\item Description list 1
\item Description list 2
\item Description list 3
\end{description}
\end{verbatim}

\subsection{Enunciations like theorem, lemma etc.}
\label{sec3.13}

The \AmS-\LaTeX\ package for enunciations (amsthm.sty) has been already loaded in the class file.

To get the theorem environment use the coding as:
\begin{verbatim}
\begin{theorem}
Theorem text. Theorem text. Theorem text.
Theorem text. Theorem text. Theorem text.
\end{theorem}
\end{verbatim}

and \verb|\newtheorem{theorem}{Theorem}| in the preamble.

Similarly, we can define for lemma, corollary, proposition, definition etc.

\subsection{Cross-referencing}
\label{sec3.14}

LATEX provides the following commands for cross referencing\vskip1pc

\noindent \verb|\label{marker}|, \verb|\ref{marker}| and \verb|\pageref{marker}|\vskip1pc

\noindent where marker is an identifier chosen by the user. LATEX replaces \verb|\ref| by
the number of the section, subsection, figure, table, or theorem after which
the corresponding \verb|\label| command was issued. \verb|\pageref| prints the page
number of the page where the \verb|\label| command occurred.

\subsection{Citations}

Citations are made with the \verb|\cite| command as usual. In this class file we have
used natbib.sty for cross references and reference style.

For bibliography the natbib package has been defined in the template as \verb|\usepackage{natbib}|
with \verb|\bibpunct{[}{]}{,}{n}{,}{;}| command

For more details about natbib.sty can be found at
http://ctan.org/tex-archive/macros/latex/contrib/natbib/

\section{Back Matter}

\begin{verbatim}

\begin{Backmatter}

\section*{Acknowledgments}

Acknowledgments


\begin{thebibliography}{}

...
..
\end{thebibliography}

\appendix
\section*{Appendix}

\end{Backmatter}

\end{verbatim}

\begin{Backmatter}

\section*{Acknowledgements}

Acknowledgements and other unnumbered sections are created
using the \verb|\section*| command:

\subsection{References}
\label{sec3.18}

The reference entries can be \LaTeX\ typed bibliographies or generated through a BIB\TeX\ database.
BIB\TeX\ is an adjunct to \LaTeX\ that aids in the preparation of bibliographies. BIB\TeX\
allows authors to build up a database or collection of bibliography entries that may be used for many
manuscripts. They also save us the trouble of having to specify formatting. More details can be found
in the \textit{BIB\TeX\ Guide}. For \LaTeX\ reference entries use the
\verb|\begin{thebibliography}...\end{thebibliography}| environment (see below) to make references in your paper.
By default the class file will produce the numbered \LaTeX\ bibliography.

\begin{verbatim}
\begin{thebibliography}{}
\bibitem[Cadero et al.(2018)]{cadero2018global}
{Cad\'ero, A., Aubry, A., Brun, F.,   Dourmad, J. Y., Sala\'l\'zn, Y.
 and Garcia-Launay, F.} (2018).
Global sensitivity analysis of a pig fattening unit model simulating
technico-economic performance and environmental impacts.
\textit{Agricultural Systems}, {165}, 221--229.

\bibitem[Cao(2015)]{r16}
{Cao, L.} (2015). Improved Genetic Algorithm for Fast Path Planning of USV.
\textit{International Symposium on Multispectral Image Processing
 and Pattern Recognition (MIPPR2015)}, 9815, 981529.

\bibitem[Cheng et al.(2015)]{r27}
{Cheng, Z., Tong, Y., Shen L. and Ming, L. I.} (2015). Improved bacteria
foraging optimisation algorithm for solving flexible job-shop scheduling problem.
\textit{Journal of Computer Applications}, 63--67.

\end{thebibliography}

\end{verbatim}


\subsection{Formatting}
\label{sec3.8}

One should always use \LaTeX\ macros rather than the lower-level
\TeX\ macros like \verb|\it|, \verb|\bf| and \verb|\tt|. The
\LaTeX\ macros offer much improved features. The following table summarizes the font
selection commands in \LaTeX.

\subsection*{\LaTeX\ text formatting commands}
\begin{tabular}{ll@{\hskip60pt}ll}
\verb|\textit|  & Italics      &\verb|\textsf|  & Sans Serif\\[3pt]
\verb|\textbf|  & Boldface     &\verb|\textsc|  & Small Caps\\[3pt]
\verb|\texttt|  & Typewriter   &\verb|\textmd|  & Medium Series\\[3pt]
\verb|\textrm|  & Roman        &\verb|\textnormal| & Normal Series\\[3pt]
\verb|\textsl|  & Slanted      &\verb|\textup|  & Upright Series
\end{tabular}


\subsection*{\LaTeX\ math formatting commands}
\begin{tabular}{ll@{\qquad}ll}
\verb|\mathit|     & Math Italics        &\verb|\mathfrak|   & Fraktur\\[3pt]
\verb|\mathbf|     & Math Boldface       &\verb|\mathbb|     & Blackboard Bold\\[3pt]
\verb|\mathtt|     & Math Typewriter     &\verb|\mathnormal| & Math Normal\\[3pt]
\verb|\mathsf|     & Math Sans Serif     &\verb|\boldsymbol| & Bold math for Greek letters\\[3pt]
\verb|\mathcal|    & Calligraphic        &                   & and other symbols
\end{tabular}

\newpage

\section{\color{dscolor}Macro packages}
\label{sec4}

The commonly used packages which can be used frequently are:

\begin{verbatim}
amsmath      graphicx     rotating         multirow
amssymb      endnotes     subfigure        tikz
amsfonts     setspace     array            siunitx
xspace       latexsym     url              natbib
amscd        multicol     algorithm        biblatex
\end{verbatim}


Additionally, you can use other packages and these should be loaded
using the \verb|\usepackage| command in the preamble.

\appendix
\section*{\color{dscolor}Appendix}
\label{sec3.16}

The \verb|\appendix| command signals that all following sections are
appendices, and therefore the headings after \verb|\appendix| will be set
as appendix headings.

Note: All the figures, tables, equations, enunciations will be automatically
numbered as A.1, A.2, etc. in the appendix part.
\end{Backmatter}

\end{document}
